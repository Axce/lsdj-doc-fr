\chapter{Premiers pas}


\section{Le son de la Game Boy}

La puce sonore de la Game Boy dispose de quatre canaux, chacun avec une résolution de 4 bits.

\begin{description}
\item[Canal Pulse 1] Onde carrée avec fonctions d’enveloppe et de balayage (sweep).
\item[Canal Pulse 2] Onde carrée avec fonction d’enveloppe.
\item[Canal Wave] Synthétiseur, lecture de samples et synthèse vocale logiciels.
\item[Canal Noise] Bruit avec fonctions d’enveloppe et de forme.
\end{description}

\section{Touches}

Dans cette documentation, les touches sont indiquées de la manière suivante :
\begin{description}
\item[\textsc{a}] bouton \textsc{a}
\item[\textsc{b}] bouton \textsc{b}
\item[\textsc{start}] bouton \textsc{start}
\item[\textsc{select}] bouton \textsc{select}
\item[\textsc{cursor}] toute direction de la croix directionnelle
\item[\textsc{left}] croix directionnelle gauche
\item[\textsc{right}] croix directionnelle droite
\item[\textsc{up}] croix directionnelle haut
\item[\textsc{down}] croix directionnelle bas
\item[\textsc{left/right}] croix directionnelle gauche ou droite
\item[\textsc{up/down}] croix directionnelle haut ou bas
\item[\textsc{select+a}] appuyer sur \textsc{a} en maintenant \textsc{select}
\item[\textsc{select+(b,b)}] appuyer deux fois sur \textsc{b}, en maintenant \textsc{select}
\end{description}

\section{Navigation dans le programme}

\begin{figure}[hbtp]
\centering
\fbox{ \includegraphics{song} }
\caption{Écran \textsc{song}}
\label{fig:song}
\end{figure}

Little Sound Dj démarre sur l’écran \textsc{song}.
Les quatre colonnes \textsc{pu1, pu2, wav} et \textsc{noi} représentent les quatre canaux sonores de la Game Boy.
Il y a deux canaux à onde carrée, un canal à onde personnalisée et un canal de bruit.
Utilisez la croix directionnelle pour déplacer le curseur d’un canal à l’autre.

\begin{figure}[hbtp]
\centering
\fbox{ \includegraphics[width=2.5cm]{map} }
\caption{Carte des écrans}
\label{fig:map}
\end{figure}

Little Sound Dj comporte neuf écrans, disposés selon une carte affichée en bas à droite de l’écran (figure~\ref{fig:map}).

Les écrans les plus utilisés se trouvent sur la rangée du milieu, classés par niveau de détail.
Les écrans \textsc{song}, \textsc{chain} et \textsc{phrase} servent à la composition et fonctionnent en structure arborescente :
le morceau (\textsc{song}) contient des chaînes (\textsc{chains}),
chaque chaîne contient des phrases (\textsc{phrases}),
et chaque phrase contient des notes.
Ils sont suivis des écrans \textsc{instrument} et \textsc{table}, utilisés pour créer des sons.

Pour passer d’un écran à l’autre, appuyez sur \textsc{select+cursor}.

\section{Aide intégrée}

\begin{figure}[hbtp]
\centering
\fbox{\includegraphics{findhelp}}
\caption{Aide de l’écran \textsc{project}}
\end{figure}

Pour accéder à l’aide intégrée, allez sur l’écran \textsc{project} (au-dessus de l’écran \textsc{song}) et appuyez sur \textsc{a} sur \textsc{help}.
L’écran d’aide répertorie les combinaisons de touches pour les différents écrans, ainsi qu’une liste de commandes.

\section{Créer vos premiers sons}

Allez sur l’écran \textsc{song}, placez le curseur sur la colonne \textsc{pu1}, et appuyez sur \textsc{a}.
Une nouvelle chaîne « 00 » apparaît.
Éditez-la en appuyant sur \textsc{select+right} pour ouvrir l’écran \textsc{chain}.
Là, appuyez sur \textsc{a} pour insérer une nouvelle phrase, puis \textsc{select+right} pour passer à l’écran \textsc{phrase}.

\begin{figure}[hbtp]
\centering
\fbox{\includegraphics{phrase}}
\caption{Écran \textsc{phrase}}
\label{fig:phrase1}
\end{figure}

Dans cet écran, vous pouvez saisir des notes à jouer.
Placez le curseur dans la colonne \textsc{note} et appuyez sur \textsc{a} : le texte C-2 apparaît : C étant la note (do), et 2 l’octave.
Appuyez sur \textsc{start} pour lire la phrase.
Remarquez que la phrase est lue de haut en bas.
Vous pouvez modifier la note en maintenant \textsc{a} et en utilisant la croix directionnelle :
\textsc{a+left/right} change la note, et \textsc{a+up/down} change l’octave.

Maintenant, déplacez le curseur et insérez d’autres notes si vous le souhaitez.
Pour supprimer une note, appuyez sur \textsc{a} en maintenant \textsc{b}.
Quand vous avez fini d’écouter, appuyez de nouveau sur \textsc{start} pour arrêter la phrase.

Le son de base du canal Pulse peut vite devenir monotone. Passons donc à l’écran \textsc{instrument} avec \textsc{select+right}.

\begin{figure}[hbtp]
\centering
\fbox{\includegraphics{instr-pulse}}
\caption{Écran « Instrument »}
\label{fig:instr}
\end{figure}

Dans l’écran des instruments, on peut rendre le son plus intéressant.
Essayez de modifier les champs \textsc{env.} et \textsc{wave} en déplaçant le curseur dessus, puis en appuyant sur \textsc{a+left/right}.
Par exemple, régler \textsc{env.} sur \texttt{83} rendra le son plus court.
Appuyez sur \textsc{start} pour écouter les changements en temps réel !

Le champ \textsc{type} définit le type d’instrument.
Les types sont propres à chaque canal :
les instruments \textsc{pulse} ne fonctionnent que dans les canaux à onde carrée,
les instruments \textsc{wave} et \textsc{kit} dans le canal wave,
et les instruments \textsc{noise} dans le canal de bruit.

Essayons maintenant les kits de percussions échantillonnés.
Pour cela, revenez à l’écran \textsc{Song}, placez le curseur sur le canal \textsc{Wave}, et créez une nouvelle chaîne et une nouvelle phrase en appuyant sur \textsc{a}.
Insérez une note avec \textsc{a}, puis allez éditer l’instrument avec \textsc{select+right}.
Changez le type en \textsc{kit} avec \textsc{a+right} dans le champ \textsc{type}, puis revenez à l’écran \textsc{Phrase}.
Vous pouvez désormais saisir des sons de batterie comme vous saisissiez des notes.

Pour créer de nouvelles chaînes et phrases, placez le curseur sur une ligne vide dans les écrans \textsc{Song} ou \textsc{Chain} et appuyez deux fois sur \textsc{a}.

\section{Système de numération hexadécimal}

Little Sound Dj utilise des nombres hexadécimaux.
Alors que le système décimal habituel utilise dix chiffres (0–9), le système hexadécimal en utilise seize : les chiffres 0 à 9, suivis des lettres A à F.

Par exemple, comparons les deux systèmes :

\begin{figure}[hbtp]
\centering

\begin{tabular}{r|r|r|r|r|r|r|r|r|r|r}
Décimal & 1 & 2 & 3 & 4 & 5 & 6 & 7 & 8 & 9 & 10 \\
\hline
Hexadécimal & 1 & 2 & 3 & 4 & 5 & 6 & 7 & 8 & 9 & A \\
\end{tabular}

\begin{tabular}{r|r|r|r|r|r|r|r|r|r|r}
Décimal & 11 & 12 & 13 & 14 & 15 & 16 & 17 & 18 & 19 & 20 \\
\hline
Hexadécimal & B & C & D & E & F & 10 & 11 & 12 & 13 & 14  \\
\end{tabular}

\end{figure}

Notez que les valeurs décimales et hexadécimales sont identiques ; seule la représentation change.
L’intérêt principal de l’hexadécimal est de gagner de la place à l’écran : chaque octet (valeur 0 à 255) peut être représenté avec seulement deux chiffres, de 00 à FF.

Représenter des nombres négatifs sur deux chiffres peut être délicat.
Dans Little Sound Dj, les nombres sont cycliques : si l’on soustrait 1 à 0, on obtient FF.
Ainsi, FF peut représenter –1 ou 255 selon le contexte.

Si cela semble abstrait, ne vous inquiétez pas : ce sera plus clair à l’usage.

\section{Dépannage de la cartouche}

Votre cartouche ne démarre pas, plante ou agit bizarrement ? Voici quelques pistes :

\begin{itemize}
\item Pour éviter toute perte de données, utilisez toujours des piles neuves. Remplacez-les dès que la LED rouge faiblit ou que l’écran s’assombrit.
\item Nettoyez les contacts de la cartouche avec un coton-tige imbibé d’alcool.
\item Réinsérez la cartouche plusieurs fois pour enlever l’oxyde.
\item Assurez-vous qu’elle est bien enfoncée ; un morceau de ruban adhésif peut parfois aider à la caler fermement.
\item Pour réinitialiser complètement la mémoire de la cartouche, maintenez \textsc{select+a+b} sur le bouton \textsc{load/save file} de l’écran \textsc{Project}.
\item Certaines cartouches Game Boy Advance/Nintendo DS ne sont pas compatibles avec Little Sound Dj. Si c’est le cas, essayez une version de Little Sound Dj appelée « Goomba ».
\item Consultez le wiki Little Sound Dj (\url{https://wiki.littlesounddj.com}) ou posez vos questions sur le groupe Facebook.
\end{itemize}
