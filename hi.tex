\chapter*{Salut~!}
Tout d'abord, merci d'essayer Little Sound Dj~!

Des années d'efforts ont été consacrées à rendre ce programme aussi puissant et efficace que possible. J'espère que vous l'apprécierez.

Si vous n'avez pas d'expérience préalable avec des éditeurs de musique de type ``tracker'', la quantité de nouveaux concepts peut sembler un peu écrasante au début. Ne vous inquiétez pas. Apprenez étape par étape, gardez le plaisir et progressez à votre rythme. En quelques jours, vous devriez en savoir assez sur le programme pour créer vos premières compositions.

Concernant la structure de ce manuel~: le chapitre 1 vous aide à démarrer avec les bases, tandis que le chapitre 2 propose une vue d'ensemble du programme. Ces deux premiers chapitres devraient suffire pour utiliser Little Sound Dj avec succès. Les chapitres suivants traitent de sujets avancés ou servent de référence, et peuvent être lus à votre convenance, ou ignorés complètement si vous le souhaitez.

Il y a beaucoup d'informations qui ne peuvent pas tenir dans un manuel comme celui-ci. Pour plus de ressources, consultez le Wiki maintenu par les utilisateurs à l'adresse \url{https://wiki.littlesounddj.com} --- il contient des tutoriels, des astuces et des projets hardware DIY. De plus, le groupe Facebook à l'adresse \url{https://www.facebook.com/groups/LittleSoundDJ/} est utile pour entrer en contact avec d'autres utilisateurs.

Si vous avez des questions, des idées ou des bugs à signaler concernant Little Sound Dj ou ce manuel, je serais ravi que vous me contactiez par e-mail à \href{mailto:info@littlesounddj.com}{info@littlesounddj.com}.

Bon tracking~!

\textit{/Johan}

